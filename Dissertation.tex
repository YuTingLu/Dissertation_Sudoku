% Yu-Ting Lu 24/7/2011 (Swansea)

\documentclass[11pt]{report}

\usepackage{sudoku}


\begin{document}

\title{Sudoku Patterns}
\author{Yu-Ting Lu\\
 Computer Science of Swansea University,\\
 461467@swansea.ac.uk}
\date{July 2011}
\maketitle

\tableofcontents



\chapter{Introduction}
\label{cha:Introduction}


Sudoku is a puzzle which is popularised into the world for ages. And there are amount of people spend times on either hard puzzle generated, or looking for the best techniques to improve the efficiency on solving difficult puzzle.
When the puzzle difficulty becomes harder and harder, then more advanced techniques will be required to be used together or in order to cope the puzzle. Surely, as the difficulties level goes up, the cost of time raise up even thought the suitable technique is given to help. Consequently, in this article, the first chapter is going to know the background of Sudoku and what is Sudoku, and the secondly understand discussed after understanding how these special techniques applied on eliminating impossibilities in puzzle solving. Afterwards, the efficiency of the special technique for solving Sudoku will be the final topic to discuss about in the end.
In this dissertation, the goal to achieve is to experience the time of applying special techniques on solving Sudoku puzzles. In addition, to find out the most efficient way to apply the combination of more than techniques and how many puzzles could it achieve to solve.


\section{What is Sudoku?}
\label{sec:whatissudoku}

Sudoku uses an $N \times N$ matrix, where $N = n \cdot n$, and for standard Sudoku we have $n = 3$, thus $N = 9$. Each entry (``cell'') contains a number in $\{1, \dots, N\}$, or it might be empty, that is, to be filled out. A completed Sudoku puzzle must fulfil the following additional requirements:
\begin{enumerate}
\item In every row and in every column every number occurs only (exactly) once.
\item The same is true for the ``blocks'':
  \begin{enumerate}
  \item The whole matrix is sub-divided into $N$ blocks.
  \item Each block is an $n \times n$ matrix.
  \end{enumerate}
  Now in each block every number has also to occur (exactly) once.
\end{enumerate}
For ``classical Sudoku'' puzzles we have the following additonal rules:
\begin{enumerate}
\item A Sudoku puzzle must have at least one solution (no unsolvable puzzles are normally considered).
\item And in fact, a Sudoku puzzle must have a unique solution (no multiple solutions are normally allowed).
\end{enumerate}
An example:

\setlength\sudokusize{8cm}
\begin{figure}
\begin{sudoku}
 |2|5| | |3| |9| |1|.
 | |1| | | |4| | | |.
 |4| |7| | | |2| |8|.
 | | |5|2| | | | | |.
 | | | | |9|8|1| | |.
 | |4| | | |3| | | |.
 | | | |3|6| | |7|2|.
 | |7| | | | | | |3|.
 |9| |3| | | |6| |4|.
\end{sudoku}
\caption{Sudoku Problem}
\end{figure}

\begin{figure}
\begin{sudoku}
  |2|5|8|7|3|6|9|4|1|.
  |6|1|9|8|2|4|3|5|7|.
  |4|3|7|9|1|5|2|6|8|.
  |3|9|5|2|7|1|4|8|6|.
  |7|6|2|4|9|8|1|3|5|.
  |8|4|1|6|5|3|7|2|9|.
  |1|8|4|3|6|9|5|7|2|.
  |5|7|6|1|4|2|8|9|3|.
  |9|2|3|5|8|7|6|1|4|.
\end{sudoku}
\caption{Sudoku Solution}
\end{figure}





\section{Literature}
\label{sec:introLiterature}

All the cells of Sudoku puzzle could be re-permutated in sequence to form an extended Sudoku board with the original one.
example:
\begin{sudoku}
   | | | | | | | |1|2|.
   | | | | | | | | | |.
   | | | | | | | | | |.
   | | | | | | | | | |.
   | | | | | | | | | |.
   | | | | | | | | | |.
   | | | | | | | | | |.
   | | | | | | | | | |.
   | | | | | | | | | |.
\end{sudoku}
could be re-permutated to get an extended one.
\begin{sudoku}
\label{p48}
   |8|9| | | | | | |.
   | | | | | | | | | |.
   | | | | | | | | | |.
   | | | | | | | | | |.
   | | | | | | | | | |.
   | | | | | | | | | |.
   | | | | | | | | | |.
   | | | | | | | | | |.
   | | | | | | | | | |.
\end{sudoku}

\cite{Berthier2007Sudoku} XXX

\chapter{Techniques for solving Sudoku}
\label{sec:Techniques}

Sudoku problems have classified into different level according to its difficulties. To solve the puzzles, there are loads of techniques could be used to support user to solve Sudoku puzzles in different level of difficulties.
These techniques are usually divided into two main parts, one is called “Direct Elimination Technique”, and the other is named “Candidates Elimination Techniques".
Direct Elimination Technique is an approach could be easily applied without any pencilmarks written on. It eliminates the impossibilities through analysing the existing numbers given in the question step by step. It is the most common and easy way to cope the puzzles, but it will work insufficiently if the harder question has been encountered. 
Consequently, Candidates Elimination Techniques are provided to solve harder question. To apply this skill to deal Sudoku question, pencilmarks are mostly needed to be written in all the grids at the first stage. It is a step to write all the possibilities in all the awaiting grids. Pencilmarks will help with skills applied to eliminate all the impossibilities till the last unique digit is sure for players to fill the right digit in.
Generally, most of the Sudoku problem which is classified to easy level could be solved by applying naked single and hidden single skill if the Direct Elimination Technique is inapplicable. According to the Degree of difficulty, more and more Candidates Elimination Techniques will need to be applied jointly to figure out the solution. 
Naked single and hidden single are the basic methods of Candidates Elimination Techniques mostly used to start the game. When the harder puzzle comes, the techniques: naked pair, hidden pair, naked triplet, hidden triplet, naked quad or even hidden quad techniques will involves in solving the puzzle. Usually the general puzzles could be deal by the techniques introduced lately. X-wing, XY-wing, XYZ-wing, WXYZ-wing, and Swordfish are rare used only if the infrequent problem of high level difficulty appears. 

\section{Naked Single}
\label{sec:Naked Single}

Naked single is the easiest method to start with, which, could be formed automatically after the pencilmarks are fully written in the grids. The digit could be easily decided to write into the specific grid of the puzzle.

\section{Naked Pair}
\label{sec:Naked Pair}

Naked Pair is defined in the situation that 2 grids have 2 same candidate digits in both 2 specific grids which exactly in the same row, same column, or same box. By applying this skill, some grids could eliminate those has same digit possibilities as naked pair, because the possibilities is sure to only apply in the selected naked pair after the examination by this skill.

\section{Naked Triplet}
\label{sec:Naked Triplet}
It is in the situation whilst three grids all in the same row, same column, or same box and have common 3 possibilities in these 3 grids. Thus, the result that could be inferred is, those related rows, columns, and the block could remove the numbers which has same the possibilities in as these three grids.

\section{Naked Quad}
\label{sec:Naked Quad}
It is in the situation while four grids in the same row, same column, or same box are having common 4 possibilities digits in. Thus, the result which could be inferred is those related rows, columns, and the block could remove the numbers which as same candidates in as that four grids.

\section{Hidden Single}
\label{sec:Hidden Single}
Hidden single does not show the solution obviously like what naked single does (showing one candidate only in the grid). It is a skill to find a unique specific digit to write in only if the player looks closely. However, it sometimes replaced by Cross-Hatching which could be even more easily applied to find the answer without the pencilmarks step. 

\section{Hidden Pair}
\label{sec:Hidden Pair}
The definition is, for Hidden Pair, that 2 possible candidates are only restricted to 2 grids in a common row, a common column, or a common box. 

\section{Hidden Triplet}
\label{sec:Hidden Triplet}
Basically, it is evolved from hidden single and hidden pair. The definition is 3 possible candidates are restricting to 3 grids in a row, a column, or a box only.

\section{Hidden Quad}
\label{sec:Hidden Quad}
This is the circumstance which has four digits candidate found to be restricted to specific 4 cells which has equal symmetry. Then other possibilities digits in these four cells should be removed afterwards. It is aims to help player to eliminate impossibilities in the grids by hidden quad skill. 

\section{X-wing}
\label{sec:X-wing}
It is an advanced technique to apply on when 4 grids are fully symmetric and have 2 candidates for a given digit in the symmetric grids of parallel rows. All other candidates for that digit from these 2 columns could be eliminated. On the other hand, all other candidates for that digit from these 2 rows could be removed if the situation is reversed.

\section{XY-wing}
\label{sec:XY-wing}

\section{XYZ-wing}
\label{sec:XYZ-wing}

\section{WXYZ-wing}
\label{sec:WXYZ-wing}

\section{Swordfish}
\label{sec:Swordfish}
Swordfish is an extended approach for X-wing which has really rare opportunity to apply on. It uses when all candidates for specific digit X are occurred in only three columns, all candidates for digit X not in these three columns could be eliminated. Each defined rows could have 2 or 3 candidates for specific digit X.

\section{Intersection Removal}
\label{sec:Intersection Removal}
Sometimes, only hidden single and naked single could be able to deal all the Sudoku problems. When more difficult problems have been met, player will need to have more skill to support.
The intersection removal technique provides player to eliminate the impossibilities out of the row, or the column which related to the object box.
Briefly, the digit which is being concentrated on is only occurred in the same row or same column of the object box. Extending the view to complete row (or complete column) linked of whole box to eliminate all its possibilities in other boxes.




\chapter{Conclusion}
\label{sec:Conclusion}
The Sudoku puzzle might be solved by only one technique in some simple case Nevertheless, usually two or more are commonly needed in dealing general problem such like using both naked single and hidden single. The number of techniques used will increase according to the difficulty of the puzzle. The harder problem needs more complex technique to support it.
These special approaches for solving puzzle are simple and in fact popularly used in general problem solving. They are efficient to support player to eliminate the impossibilities in the grids, and leave the rest possibilities to get to the final solution.


\bibliographystyle{plain}

\bibliography{Sudoku}

\end{document}
