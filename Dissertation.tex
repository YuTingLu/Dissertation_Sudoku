% Yu-Ting Lu 24/7/2011 (Swansea)

\documentclass[11pt]{report}

\usepackage{sudoku}

\newcommand{\cell}[9]{%
  \scriptsize
  \setlength{\tabcolsep}{1pt}
  \renewcommand{\arraystretch}{0.5}
  \hspace{-0.6em}
  \begin{tabular}{ccc}
    #1 & #2 & #3\\
    #4 & #5 & #6\\
    #7 & #8 & #9
  \end{tabular}
}

\begin{document}

\title{Sudoku Patterns}
\author{Yu-Ting Lu\\
 Computer Science of Swansea University,\\
 461467@swansea.ac.uk}
\date{July 2011}
\maketitle

\tableofcontents



\chapter{Introduction}
\label{cha:Introduction}


Sudoku is a puzzle which is popularised into the world for ages. And there are amount of people spend times on either hard puzzle generated, or looking for the best techniques to improve the efficiency on solving difficult puzzle.
When the puzzle difficulty becomes harder and harder, then more advanced techniques will be required to be used together or in order to cope the puzzle. Surely, as the difficulties level goes up, the cost of time raise up even thought the suitable technique is given to help. Consequently, in this article, the first chapter is going to know the background of Sudoku and what is Sudoku, and the secondly understand discussed after understanding how these special techniques applied on eliminating impossibilities in puzzle solving. Afterwards, the efficiency of the special technique for solving Sudoku will be the final topic to discuss about in the end.
In this dissertation, the goal to achieve is to experience the time of applying special techniques on solving Sudoku puzzles. In addition, to find out the most efficient way to apply the combination of more than techniques and how many puzzles could it achieve to solve.


\section{What is Sudoku?}
\label{sec:whatissudoku}

Sudoku uses an $N \times N$ matrix, where $N = n \cdot n$, and for standard Sudoku we have $n = 3$, thus $N = 9$. Each entry (``cell'') contains a number in $\{1, \dots, N\}$, or it might be empty, that is, to be filled out. A completed Sudoku puzzle must fulfil the following additional requirements:
\begin{enumerate}
\item In every row and in every column every number occurs only (exactly) once.
\item The same is true for the ``blocks'':
  \begin{enumerate}
  \item The whole matrix is sub-divided into $N$ blocks.
  \item Each block is an $n \times n$ matrix.
  \end{enumerate}
  Now in each block every number has also to occur (exactly) once.
\end{enumerate}
For ``classical Sudoku'' puzzles we have the following additonal rules:
\begin{enumerate}
\item A Sudoku puzzle must have at least one solution (no unsolvable puzzles are normally considered).
\item And in fact, a Sudoku puzzle must have a unique solution (no multiple solutions are normally allowed).
\end{enumerate}
An example:

\setlength\sudokusize{8cm}
\begin{figure}
\begin{sudoku}
 |2|5| | |3| |9| |1|.
 | |1| | | |4| | | |.
 |4| |7| | | |2| |8|.
 | | |5|2| | | | | |.
 | | | | |9|8|1| | |.
 | |4| | | |3| | | |.
 | | | |3|6| | |7|2|.
 | |7| | | | | | |3|.
 |9| |3| | | |6| |4|.
\end{sudoku}
\caption{Sudoku Problem}
\end{figure}

\begin{figure}
\begin{sudoku}
  |2|5|8|7|3|6|9|4|1|.
  |6|1|9|8|2|4|3|5|7|.
  |4|3|7|9|1|5|2|6|8|.
  |3|9|5|2|7|1|4|8|6|.
  |7|6|2|4|9|8|1|3|5|.
  |8|4|1|6|5|3|7|2|9|.
  |1|8|4|3|6|9|5|7|2|.
  |5|7|6|1|4|2|8|9|3|.
  |9|2|3|5|8|7|6|1|4|.
\end{sudoku}
\caption{Sudoku Solution}
\end{figure}





\section{Literature}
\label{sec:introLiterature}

Systematic procedure for solving a puzzle:
\begin{tabbing}
Loop Until \= a solution is found (or until it is proven there can be no solution) \\
\> Do until \= a rule applies  effectively \\
\> \> Take the first rule not yet tried in the list \\
\> \> Do until \= its conditions pattern effectively maps to the cell \\
\> \> \> Try all possible mappings of the conditions pattern \\
\> \> End do\\
\> End do\\
\> Apply rule on selected matching pattern \\
End loop\\
\end{tabbing}
All the cells of Sudoku puzzle could be re-permutated in sequence to form an extended Sudoku board with the original one.

New representations of a puzzle :
\begin{enumerate}
  \item number n is in rc-cell (r, c),
  \item column c is in rn-cell (r, n),
  \item row r is in cn-cell (c, n),
  \item square s is in bn-cell (b, n) - where (r, c) = [b, s].
\end{enumerate}

example:

could be re-permutated to get an extended one.

\cite{Berthier2007Sudoku} XXX





\section{Basic notions and notations}
\label{sec:basicnotnotat}
To begin introducing the techniques for solving Sudoku, there are few basic notions that have to be defined and declared.
A Sudoku problem which has the format like the matrix below is called a  ``puzzle'' here.

\begin{sudoku}
\label{p48}
   |1|{\cell {}{}3{}56{}{}{}}|{\cell {}{}3{}{}678{}}|2|{\cell {}{}{}{}{}6{}89}|{\cell {}{}{}{}{}{}{}8{}}|{\cell {}{}{}{}5{}{}89}|{\cell {}{}3{}{}{}{}89}|4|.
   |{\cell {}2{}{}{}{}{}8{}}|9|{\cell {}{}{}{}{}{}{}8{}}|5|{\cell {}{}{}4{}{}{}8{}}|3|6| 7|{\cell 12{}{}{}{}{}8{}}|.
   |{\cell {}23{}56{}8{}}|4|{\cell {}{}3{}{}6{}8{}}|{\cell 1{}{}{}{}6{}89}|7|{\cell 1{}{}{}{}{}{}8{}}|{\cell 12{}{}5{}{}89}|{\cell {}{}3{}{}{}{}89}|{\cell 123{}{}{}{}8{}}|.
   |{\cell {}{}3{}{}6{}8{}}|1|{\cell {}{}34{}6{}8{}}|{\cell {}{}{}4{}678{}}|{\cell {}234{}6{}8{}}|{\cell {}2{}4{}{}78{}}|{\cell {}{}{}4{}{}78{}}|5|9|.
   |{\cell {}{}{}{}56{}8{}}|{\cell {}{}{}{}56{}{}{}}|2|{\cell 1{}{}4{}6789}|{\cell {}{}{}456{}89}|{\cell 1{}{}45{}78{}}|3|{\cell {}{}{}4{}{}{}8{}}|{\cell 1{}{}{}{}678{}}|.
   |9|7|{\cell {}{}34{}6{}8{}}|{\cell 1{}{}4{}6{}8{}}|{\cell {}{}3456{}8{}}|{\cell 1{}{}45{}{}8{}}|{\cell 1{}{}4{}{}{}8{}}|2|{\cell 1{}{}{}{}6{}8{}}|.
   |{\cell {}23{}{}{}7{}{}}|{\cell {}23{}{}{}{}{}{}}|{\cell {}{}3{}{}{}7{}9}|{\cell {}{}{}4{}{}78{}}|1|{\cell {}2{}45{}78{}}|{\cell {}2{}4{}{}789}|6|{\cell {}23{}{}{}78{}}|.
   |{\cell {}2{}{}{}67{}{}}|8|5|3| {\cell {}2{}4{}{}{}{}{}}|9| {\cell {}2{}4{}{}7{}{}}|1| {\cell {}2{}{}{}{}7{}{}}|.
   |4| {\cell {}23{}{}{}{}{}{}}|{\cell 1{}3{}{}{}7{}9}|{\cell {}{}{}{}{}{}78{}}|{\cell {}2{}{}{}{}{}8{}}|6|{\cell {}2{}{}{}{}789}|{\cell {}{}3{}{}{}{}89}| 5|.
\end{sudoku}

In Sudoku, each puzzle is divided into $N$ portions (called  ``blocks'') which is highlighted by the bold lines, and has $N (n \times n)$ sub-matrix (called ``cells'') in each block. Every cell has 9 possibilities could have candidate (or digit) $\{1, \dots, N\}$ filled in. As the figure shown above, users could have all the possibilities pencil-mark as same as the figure does to begin with techniques apply. To put in short, every block has $N$ cells in it, and there are $N$ blocks consisted in a puzzle. In the puzzle, there are totally $N \times N$ cells equally separate into $N$ blocks, columns(vertically), and rows (horizontally).

\begin{table}
  \setlength{\tabcolsep}{2.5pt}
  \renewcommand{\arraystretch}{1.6}
  \hspace{1em}
  \begin{tabular}{ | r| r| r | }
  \hline
  {\cell {}{}{}{}{}6{}{}9} &{\cell {}{}{}{}{}6{}8{}} & 1 \\ \hline
  {\cell {}2{}{}56{}8{}}  & {\cell {}2{}{}5{}789} & 3 \\ \hline
  {\cell {}{}{}{}5{}7{}9} & 4 & {\cell {}{}{}{}{}6{}{}9} \\
  \hline
  \end{tabular}
  \caption{Example for Blocks}
\end{table}

\begin{table}
  \centering
  \setlength{\tabcolsep}{2.5pt}
  \renewcommand{\arraystretch}{1.6}
  \hspace{1em}
  \begin{tabular}{ | r| r| r | r| r| r| r| r| r| }
  \hline
  {\cell {}{}{}{}{}6{}{}9} & {\cell {}{}{}{}{}6{}8{}} & 1 {\cell {}2{}{}56{}8{}}  & {\cell {}2{}{}5{}789} & 3 {\cell {}{}{}{}5{}7{}9} & 4 & {\cell {}{}{}{}{}6{}{}9} \\
  \hline
  \end{tabular}
  \caption{Example for Columns}
\end{table}

\begin{table}
  \setlength{\tabcolsep}{2.5pt}
  \renewcommand{\arraystretch}{1.6}
  \hspace{1em}

\begin{tabular}{ | r| }
\hline
  {\cell {}{}{}{}{}6{}{}9} \\ \hline
  {\cell {}{}{}{}{}6{}8{}}\\ \hline
   1 \\ \hline
  {\cell {}2{}{}56{}8{}} \\ \hline
  {\cell {}2{}{}5{}789} \\ \hline
  3 \\  \hline
  {\cell {}{}{}{}5{}7{}9} \\ \hline
  4 \\ \hline
 {\cell {}{}{}{}{}6{}{}9} \\
\hline
\end{tabular}
\caption{Example for Rows}
\end{table}


when the candidate must be eliminated after rules applied, then this specific candidate must remove simultaneously from three representations.
\begin{enumerate}
  \item eliminate n from rc-cell (r, c) in the standard rc-representation,
  \item eliminate c from rn-cell (r, n) in the rn-representation,
  \item eliminate r from cn-cell (c, n) in the cn-representation;
\end{enumerate}
As same, when a certain value has to insert to cell, update has to be done simultaneously in all three representations.
\begin{enumerate}
  \item insert n as the value of rc-cell (r, c) in the standard rc-representation,
  \item insert c as the value of rn-cell (r, n) in the rn-representation,
  \item insert r as the value of cn-cell (c, n) in the cn-representation.
\end{enumerate}

XXX how to speak about a ``Sudoku problem''; as in the book, with additional explanations and examples XXX

Notations relating to a specific instance: XXX explaining sub-matrices XXX








\chapter{Techniques for solving Sudoku}
\label{sec:Techniques}

Sudoku problems have classified into different level according to its difficulties. To solve the puzzles, there are loads of techniques could be used to support user to solve Sudoku puzzles in different level of difficulties.
These techniques are usually divided into two main parts, one is called “Direct Elimination Technique”, and the other is named “Candidates Elimination Techniques".
Direct Elimination Technique is an approach could be easily applied without any pencilmarks written on. It eliminates the impossibilities through analysing the existing numbers given in the question step by step. It is the most common and easy way to cope the puzzles, but it will work insufficiently if the harder question has been encountered. 
Consequently, Candidates Elimination Techniques are provided to solve harder question. To apply this skill to deal Sudoku question, pencilmarks are mostly needed to be written in all the cells at the first stage. It is a step to write all the possibilities in all the awaiting cells. Pencilmarks will help with skills applied to eliminate all the impossibilities till the last unique digit is sure for players to fill the right digit in.
Generally, most of the Sudoku problem which is classified to easy level could be solved by applying naked single and hidden single skill if the Direct Elimination Technique is inapplicable. According to the Degree of difficulty, more and more Candidates Elimination Techniques will need to be applied jointly to figure out the solution. 
Naked single and hidden single are the basic methods of Candidates Elimination Techniques mostly used to start the game. When the harder puzzle comes, the techniques: naked pair, hidden pair, naked triplet, hidden triplet, naked quad or even hidden quad techniques will involves in solving the puzzle. Usually the general puzzles could be deal by the techniques introduced lately. X-wing, XY-wing, XYZ-wing, WXYZ-wing, and Swordfish are rare used only if the infrequent problem of high level difficulty appears. 


\section{Naked Single}
\label{sec:Naked Single}

Naked single is the easiest method that user could put whose hands on with. The situation to apply naked single, firstly, all the possible candidates would be pencilmarked in all the empty cells. After the work is done for all the cells. Look into the puzzle, find if there are any cells has one possibility only. If so, then, yes it is the situation what we called ``Naked Single'', and candidate found is the only solution to fill into the specific cell of the puzzle.

\begin{figure}
\begin{sudoku}
\label{nakedsingle}
   |1|{\cell {}{}3{}56{}{}{}}|{\cell {}{}3{}{}678{}}|2|{\cell {}{}{}{}{}6{}89}|{\cell {}{}{}{}{}{}{}8{}}|{\cell {}{}{}{}5{}{}89}|{\cell {}{}3{}{}{}{}89}|4|.
   |{\cell {}2{}{}{}{}{}8{}}|9|{\cell {}{}{}{}{}{}{}8{}}|5|{\cell {}{}{}4{}{}{}8{}}|3|6| 7|{\cell 12{}{}{}{}{}8{}}|.
   |{\cell {}23{}56{}8{}}|4|{\cell {}{}3{}{}6{}8{}}|{\cell 1{}{}{}{}6{}89}|7|{\cell 1{}{}{}{}{}{}8{}}|{\cell 12{}{}5{}{}89}|{\cell {}{}3{}{}{}{}89}|{\cell 123{}{}{}{}8{}}|.
   |{\cell {}{}3{}{}6{}8{}}|1|{\cell {}{}34{}6{}8{}}|{\cell {}{}{}4{}678{}}|{\cell {}234{}6{}8{}}|{\cell {}2{}4{}{}78{}}|{\cell {}{}{}4{}{}78{}}|5|9|.
   |{\cell {}{}{}{}56{}8{}}|{\cell {}{}{}{}56{}{}{}}|2|{\cell 1{}{}4{}6789}|{\cell {}{}{}456{}89}|{\cell 1{}{}45{}78{}}|3|{\cell {}{}{}4{}{}{}8{}}|{\cell 1{}{}{}{}678{}}|.
   |9|7|{\cell {}{}34{}6{}8{}}|{\cell 1{}{}4{}6{}8{}}|{\cell {}{}3456{}8{}}|{\cell 1{}{}45{}{}8{}}|{\cell 1{}{}4{}{}{}8{}}|2|{\cell 1{}{}{}{}6{}8{}}|.
   |{\cell {}23{}{}{}7{}{}}|{\cell {}23{}{}{}{}{}{}}|{\cell {}{}3{}{}{}7{}9}|{\cell {}{}{}4{}{}78{}}|1|{\cell {}2{}45{}78{}}|{\cell {}2{}4{}{}789}|6|{\cell {}23{}{}{}78{}}|.
   |{\cell {}2{}{}{}67{}{}}|8|5|3| {\cell {}2{}4{}{}{}{}{}}|9| {\cell {}2{}4{}{}7{}{}}|1| {\cell {}2{}{}{}{}7{}{}}|.
   |4| {\cell {}23{}{}{}{}{}{}}|{\cell 1{}3{}{}{}7{}9}|{\cell {}{}{}{}{}{}78{}}|{\cell {}2{}{}{}{}{}8{}}|6|{\cell {}2{}{}{}{}789}|{\cell {}{}3{}{}{}{}89}| 5|.
\end{sudoku}
\caption{Example in Naked Single}
\end{figure}

In the Figure 2.1, candidate 8 is obviously the only one possibility to fill into that two cells, thus this puzzle could be simply started completing these two cells.

XXX if a cell-matrix contains only one candidate, then the formely empty cell can be filled with this candidate XXX

??? what to do with row/column/block constraints? are the applied automatically ???


\section{Naked Pair}
\label{sec:Naked Pair}

Naked Pair is defined in the situation that 2 cells have 2 same candidate digits in both 2 specific cells which exactly in the same row, same column, or same block. By applying this skill, some cells could eliminate those has same digit possibilities as naked pair, because the possibilities is sure to only apply in the selected naked pair after the examination by this skill.

XXX we are in a ``situation'' where two cells in a row resp.\ column resp.\ block have (exactly) two candidates left, and these two candidates coincide, then these two candidates are not candidates for the other cells in the row resp.\ column. XXX

\begin{figure}
\begin{sudoku}
\label{nakedpair}
   |{\cell {}2{}45{}{}{}9}|{\cell {}2{}{}5{}789}|{\cell {}2{}45{}{}89}|{\cell {}234{}{}7{}{}}|{\cell {}234{}67{}{}}|{\cell {}{}34{}{}7{}{}}|{\cell {}{}{}{}56{}{}{}}|1|{\cell {}{}3{}45{}9}|.
  |1|3|6|9|5|8|{\cell {}2{}{}{}{}7{}{}}|{\cell {}2{}{}{}{}7{}{}}|4|.
  |{\cell {}2{}45{}{}{}9}|{\cell {}2{}{}5{}7{}9}|{\cell {}2{}45{}{}{}9}|{\cell 1234{}{}7{}{}}|{\cell {}234{}67{}{}}|{\cell 1{}34{}{}7{}{}}|{\cell {}{}{}{}56{}{}{}}|{\cell {}{}3{}{}6{}{}9}|8|.
   |{\cell {}2{}45{}{}{}{}}|{\cell {}2{}{}5{}{}{}{}}|{\cell {}2345{}{}{}{}}|{\cell 1{}345{}78{}}|{\cell {}{}34{}78{}}|6|9|{\cell {}234{}{}7{}{}}|{\cell 1{}3{}5{}{}{}{}}|.
  |7|1|{\cell {}{}345{}{}{}9}|{\cell {}{}345{}{}{}{}}|{\cell {}{}34{}{}{}{}9}|2|8|{\cell {}{}34{}6{}{}{}}|{\cell {}{}3{}56{}{}{}}|.
  |8|6|{\cell {}2345{}{}{}9}|{\cell 1{}345{}7{}{}}|{\cell {}{}34{}{}7{}9}|{\cell 1{}345{}7{}9}|{\cell 12{}45{}7{}{}}|{\cell {}234{}{}7{}{}}|{\cell 1{}3{}5{}{}{}{}}|.
  |3|{\cell {}{}{}{}5{}{}89}|{\cell {}{}{}{}5{}{}89}|{\cell {}{}{}45{}78{}}|1|{\cell {}{}{}45{}7{}9}|{\cell {}{}{}4{}6{}{}{}}|{\cell {}{}{}4{}6{}89}|2|.
  |6|4|1|{\cell {}2{}{}5{}{}8{}}|{\cell {}2{}{}{}{}{}89}|{\cell {}{}{}{}5{}{}{}9}|3|{\cell {}{}{}{}{}{}{}89}|7|.
  |{\cell {}2{}{}{}{}{}{}9}|{\cell {}2{}{}{}{}{}89}|7|6|{\cell {}234{}{}{}89}|{\cell {}{}34{}{}{}{}9}|{\cell 1{}{}4{}{}{}{}{}}|5|{\cell 1{}{}{}{}{}{}{}9}|.
\end{sudoku}
\caption{Example in Naked Pair}
\end{figure}
In the Figures 2 which is shown above, candidates 5 and 6 which highlighted by yellow colour are the only two candidates in these two cells, which is formed a naked pair. A naked pair here could help to eliminate the impossibilities for candidates 5 and 6 in other related rows, columns, and block. So those numbers which highlighted in pink will be removed from the cells after applying this skill.


\section{Naked Triplet}
\label{sec:Naked Triplet}
It is in the situation whilst three cells all in the same row, same column, or same block and have common 3 possibilities in these 3 cells. Thus, the result that could be inferred is, those related rows, columns, and the block could remove the candidates which has same the possibilities in as these three cells.

\begin{figure}
\begin{sudoku}
\label{nakedtriplet}
|2|4|{\cell {}{}{}{}{}678{}}|{\cell {}{}{}{}{}6789}|3|{\cell {}{}{}{}5{}78{}}|{\cell {}{}{}{}{}{}7{}9}|{\cell {}{}{}{}56{}8{}}|1|.
|5|9|{\cell {}{}{}{}{}678{}}|{\cell {}{}{}4{}678{}}|1|{\cell {}{}{}4{}{}78{}}|3|2|{\cell {}{}{}{}{}6{}8{}}|.
|{\cell {}{}{}{}{}{}78{}}|{\cell 1{}3{}{}67{}{}}|{\cell 1{}3{}{}678{}}|{\cell {}{}{}{}{}6789}|2|{\cell {}{}{}{}5{}78{}}|{\cell {}{}{}{}{}{}7{}9}|{\cell {}{}{}{}56{}8{}}|4|.
|3|5|2|1|4|6|8|9|7|.
|4|{\cell {}{}{}{}{}67{}{}}|{\cell {}{}{}{}{}67{}{}}|3|8|9|5|1|2|.
|1|8|9|5|7|2|6|4|3|.
|{\cell {}{}{}{}{}{}78{}}|2|{\cell {}{}{}45{}78{}}|{\cell {}{}{}4{}{}78{}}|9|3|1|{\cell {}{}{}{}{}378{}}|{\cell {}{}{}{}56{}8{}}|.
|6|{\cell {}{}3{}{}{}7{}{}}|{\cell {}{}34{}{}78{}}|{\cell {}2{}4{}{}78{}}|5|1|{\cell {}2{}4{}{}{}{}{}}|{\cell {}{}{}{}{}{}78{}}|9|.
|9|{\cell 1{}{}{}{}{}7{}{}}|{\cell 1{}{}45{}78{}}|{\cell {}2{}4{}{}78{}}|6|{\cell {}{}{}4{}{}78{}}|{\cell {}2{}4{}{}{}{}{}}|3|{\cell {}{}{}{}5{}{}8{}}|.
\end{sudoku}
\caption{Example in Naked Triplet}
\end{figure}
For example, candidates 6, 7 and 8 are the common number which appears in the cells which has these three candidates highlighted in yellow colour. Hence, the inference is those cells have candidates 6, 7 and 8 marked in pink could erase these candidates, because they are not longer possible to be the answer to fill in.

\section{Naked Quad}
\label{sec:Naked Quad}
It is in the situation while four cells in the same row, same column, or same block are having common 4 possibilities digits in. Thus, the result which could be inferred is those related rows, columns, and the block could remove the numbers which as same candidates in as that four cells.

\begin{figure}
\begin{sudoku}
\label{nakedquad}
|{\cell 1{}{}{}5{}{}{}{}}|9|4|{\cell 1{}{}{}5{}7{}{}}|3|6|2|{\cell 1{}{}{}{}{}78{}}|{\cell 1{}{}{}{}{}{}8{}}|.
|{\cell 12{}{}{}6{}{}{}}|7|{\cell 12{}{}{}6{}8{}}|{\cell 12{}{}{}{}{}{}9}|{\cell 12{}{}{}{}{}{}9}|{\cell {}{}{}{}{}{}{}89}|4|3|5|.
|{\cell 12{}{}5{}{}{}{}}|3|{\cell 12{}{}5{}{}8{}}|4|{\cell 12{}{}{}{}7{}{}}|{\cell {}{}{}{}5{}{}8{}}|{\cell {}{}{}{}67{}{}{}}|{\cell 1{}{}{}{}67{}9}|{\cell 1{}{}{}{}{}{}{}9}|.
|{\cell {}{}34{}6{}{}9}|{\cell {}{}{}4{}6{}{}{}}|{\cell {}{}3{}{}6{}{}9}|8|{\cell {}{}{}{}{}67{}9}|1|5|{\cell {}2{}{}{}{}7{}9}|{\cell {}23{}{}{}{}{}9}|.
|{\cell 1{}3{}56{}{}9}|2|{\cell 1{}3{}56{}{}9}|{\cell {}{}{}{}567{}9}|4|{\cell {}{}3{}5{}{}{}9}|{\cell {}{}{}{}{}{}78{}}|{\cell {}{}{}{}{}{}789}|{\cell {}{}3{}{}{}{}89}|.
|7|8|{\cell {}{}3{}5{}{}{}9}|{\cell {}2{}{}5{}{}{}9}|{\cell {}2{}{}{}{}{}{}9}|{\cell {}{}3{}5{}{}{}9}|1|4|6|.
|8|{\cell 1{}{}{}{}6{}{}{}}|{\cell {}23{}{}6{}{}{}}|{\cell 1{}3{}{}6{}{}{}}|5|7|9|{\cell 12{}{}{}6{}{}{}}|4|.
|{\cell {}2{}{}{}6{}{}9}|{\cell 1{}{}{}56{}{}{}}|7|{\cell 1{}{}{}{}6{}{}9}|{\cell 1{}{}{}{}6{}89}|4|3|{\cell 12{}{}56{}8{}}|{\cell 12{}{}{}{}{}8{}}|.
|{\cell {}{}34{}6{}{}9}|{\cell 1{}{}456{}{}{}}|{\cell {}{}3{}{}6{}{}9}|{\cell 1{}3{}{}6{}{}9}|{\cell 1{}{}{}{}6{}89}|2|{\cell {}{}{}{}{}6{}8{}}|{\cell 1{}{}{}56{}8{}}|7|.
\end{sudoku}
\caption{Example in Naked Quad}
\end{figure}
In the Figures 2.4 shown above, candidates 2, 3, 5, and 9 appear commonly in 4 cells in the same block. Consequently, candidates 5 and 9 which marked in pink will face to be removed because they will not be possible to be used in these cells anymore according to the inference made before. 

\section{Hidden Single}
\label{sec:Hidden Single}
Hidden single does not show the solution obviously like what naked single does (showing one candidate only in the cell). It is a skill to find a unique specific digit to write in only if the player looks closely. However, it sometimes replaced by Cross-Hatching which could be even more easily applied to find the answer without the pencilmarks step. 

\begin{figure}
\begin{sudoku}
\label{hiddensingle}
|{\cell {}23{}{}{}{}8{}}|{\cell {}234{}{}78{}}|{\cell {}{}34{}{}78{}}|{\cell {}{}{}{}{}{}78{}}|{\cell 1{}3{}{}{}7{}{}}|{\cell 123{}{}{}78{}}|5|{\cell 12{}4{}6789}|{\cell 123{}{}678{}}|.
|1|6|{\cell {}{}345{}78{}}|9|{\cell {}{}3{}{}{}7{}{}}|{\cell {}23{}5{}78{}}|{\cell {}234{}{}78{}}|{\cell {}2{}4{}{}78{}}|{\cell {}23{}{}{}78{}}|.
|{\cell {}23{}5{}{}8{}}|{\cell {}23{}5{}78{}}|9|{\cell {}{}{}{}5{}78{}}|6|4|{\cell {}23{}{}{}78{}}|{\cell 12{}{}{}{}78{}}|{\cell 123{}{}{}78{}}|.
|{\cell {}23{}56{}89}|{\cell {}23{}5{}789}|{\cell 1{}3{}5678{}}|{\cell {}{}{}{}5678{}}|{\cell 1{}{}{}{}{}7{}9}|{\cell 1{}{}{}5678{}}|{\cell {}23{}{}6789}|{\cell {}2{}{}{}6789}|4|.
|4|{\cell {}{}3{}5{}789}|{\cell {}{}3{}5678{}}|{\cell {}{}{}{}5678{}}|2|{\cell {}{}{}{}5678{}}|1|{\cell {}{}{}{}{}6789}|{\cell {}{}3{}{}678{}}|.
|{\cell {}2{}{}{}6{}89}|{\cell {}2{}{}{}{}789}|{\cell 1{}{}{}{}678{}}|3|{\cell 1{}{}4{}{}7{}9}|{\cell 1{}{}{}{}678{}}|{\cell {}2{}{}{}678{}}|5|{\cell {}2{}{}{}678{}}|.
|{\cell {}{}3{}56{}{}{}}|{\cell {}{}345{}{}{}{}}|2|{\cell {}{}{}4{}67{}{}}|8|9|{\cell {}{}{}4{}67{}{}}|{\cell 1{}{}4{}67{}{}}|{\cell 1{}{}{}567{}{}}|.
|{\cell {}{}{}{}{}6{}89}|1|{\cell {}{}{}4{}6{}8{}}|2|5|{\cell {}{}{}{}{}67{}{}}|{\cell {}{}{}4{}678{}}|3|{\cell {}{}{}{}{}678{}}|.
|7|{\cell {}{}345{}{}8{}}|{\cell {}{}3456{}8{}}|1|{\cell {}{}34{}{}{}{}{}}|{\cell {}{}3{}{}6{}{}{}}|{\cell {}2{}4{}6{}8{}}|{\cell {}2{}4{}6{}8{}}|9|.
\end{sudoku}
\caption{Example in Hidden Single}
\end{figure}
In this case, they are four possibilities could be filled into this highlighted cell. Nevertheless, if look carefully, candidate 4 is the only number that appears once only in this block. Hence, candidate 4 is the answer for this cell undoubtedly.
Actually, this cell could be solved by using either hidden single or Cross-Hatching. Looking at candidate 4 by using Cross-Hatching, all the possibilities of candidate 4 in the middle block here has only one cell left after crossing out the impossibility according to the rule that the candidate could only be used once in each row, column and the block.


\section{Hidden Pair}
\label{sec:Hidden Pair}
The definition for Hidden Pair is, that 2 possible candidates are only restricted to 2 cells in a common row, a common column, or a common block. 

\begin{figure}
\begin{sudoku}
\label{hiddenpair}
|{\cell {}234{}{}{}89}|{\cell {}{}345{}{}89}|7|{\cell {}23{}5{}{}{}9}|6|1|{\cell {}{}{}45{}{}{}9}|{\cell {}2{}{}{}{}{}89}|{\cell {}2{}4{}{}{}89}|.
|{\cell {}234{}{}{}{}9}|{\cell {}{}345{}{}{}9}|{\cell {}{}345{}{}{}9}|{\cell {}23{}5{}{}{}9}|{\cell {}2{}{}5{}{}{}9}|8|{\cell 1{}{}45{}7{}9}|6|{\cell 12{}4{}{}7{}9}|.
|1|{\cell {}{}{}{}5{}{}89}|6|{\cell {}2{}{}5{}{}{}9}|7|4|{\cell {}{}{}{}5{}{}{}9}|{\cell {}2{}{}{}{}{}89}|3|.
|{\cell {}{}{}4{}{}7{}9}|2|{\cell 1{}{}45{}{}{}9}|{\cell {}{}{}45{}{}{}9}|{\cell {}{}{}{}5{}{}{}9}|3|8|{\cell 1{}{}{}{}{}7{}9}|6|.
|{\cell {}{}34{}{}789}|{\cell 1{}3456{}89}|{\cell 1{}345{}{}{}9}|{\cell {}2{}45{}{}89}|{\cell {}2{}{}5{}{}{}9}|{\cell {}{}{}{}{}67{}{}}|{\cell 1{}34{}{}{}{}9}|{\cell 123{}{}{}7{}9}|{\cell 12{}4{}{}{}{}9}|.
|{\cell {}{}34{}{}789}|{\cell {}{}34{}6{}89}|{\cell {}{}34{}{}{}{}9}|{\cell {}2{}4{}{}{}89}|1|{\cell {}{}{}{}{}67{}{}}|{\cell {}{}34{}{}{}{}9}|5|{\cell {}2{}4{}{}{}{}9}|.
|{\cell {}{}{}4{}{}{}{}9}|{\cell 1{}{}4{}{}{}{}9}|8|7|3|2|6|{\cell 1{}{}{}{}{}{}{}9}|5|.
|6|{\cell {}{}3{}{}{}{}{}9}|2|1|8|5|{\cell {}{}3{}{}{}7{}9}|4|{\cell {}{}{}{}{}{}7{}9}|.
|5|7|{\cell 1{}3{}{}{}{}{}{}}|6|4|9|2|{\cell 1{}3{}{}{}{}8{}}|{\cell 1{}{}{}{}{}{}8{}}|.
\end{sudoku}
\caption{Example in Hidden Pair}
\end{figure}
For instance, the figure is given above. Candidates 1 and 7 which marked in yellow colour are restricted to two cells only. To put it briefly, digit number 1 and 7 could only possibly be placed in these two cells only. So the possibilities of the rest (shaded in pink) in the cells will be crossed out.


\section{Hidden Triplet}
\label{sec:Hidden Triplet}
Basically, it is evolved from hidden single and hidden pair. The definition is 3 possible candidates are restricting to 3 cells in a row, a column, or a block only.
\begin{table}
  \setlength{\tabcolsep}{2.5pt}
  \renewcommand{\arraystretch}{1.6}
  \hspace{1em}
\begin{tabular}{ | r| r| r| r| r| r| r| r| r | }
\hline
\label{hiddentriplet}
  {\cell {}{}{}{}{}6{}{}9} &{\cell {}{}{}{}{}6{}8{}} & 1 &{\cell {}2{}{}56{}8{}}  & {\cell {}2{}{}5{}789} & 3 & {\cell {}{}{}{}5{}7{}9} & 4 & {\cell {}{}{}{}{}6{}{}9} \\
\hline
\end{tabular}
\caption{Example in Hidden Triplet}
\end{table}


\section{Hidden Quad}
\label{sec:Hidden Quad}
This is the circumstance which has four digits candidate found to be restricted to specific 4 cells which has equal symmetry. Then other possibilities digits in these four cells should be removed afterwards. It is aims to help player to eliminate impossibilities in the cells by hidden quad skill. 

\begin{figure}
\begin{sudoku}
\label{hiddenquad}
|6|3|2|1|4|5|9|7|8|.
|8|1|{\cell {}{}{}{}5{}7{}{}}|{\cell {}{}3{}{}6{}{}{}}|9|{\cell {}2{}{}{}{}7{}{}}|{\cell {}23{}56{}{}{}}|{\cell {}23{}5{}{}{}{}}|4|.
|{\cell {}{}{}{}5{}{}{}9}|4|{\cell {}{}{}{}5{}7{}9}|{\cell {}{}3{}{}6{}{}{}}|8|{\cell {}2{}{}{}{}7{}{}}|{\cell {}23{}56{}{}{}}|1|{\cell {}23{}56{}{}{}}|.
|{\cell {}{}3{}{}{}{}{}9}|{\cell {}2{}{}{}{}7{}{}}|{\cell {}{}{}4{}{}{}{}9}|8|5|{\cell 1{}3{}{}{}{}{}{}}|{\cell 12{}4{}67{}{}}|{\cell {}2{}{}{}{}{}{}9}|{\cell 12{}{}{}67{}{}}|.
|1|6|{\cell {}{}{}{}5{}{}89}|2|7|4|{\cell {}{}3{}5{}{}8{}}|{\cell {}{}3{}5{}{}89}|{\cell {}{}3{}5{}{}{}{}}|.
|{\cell {}{}3{}5{}{}{}{}}|{\cell {}2{}{}{}{}7{}{}}|{\cell {}{}{}45{}{}8{}}|9|6|{\cell 1{}3{}{}{}{}{}{}}|{\cell 12{}45{}78{}}|{\cell {}2{}{}5{}{}8{}}|{\cell 12{}{}5{}7{}{}}|.
|4|8|1|5|2|9|{\cell {}{}3{}{}{}7{}{}}|6|{\cell {}{}3{}{}{}7{}{}}|.
|7|5|3|4|1|6|{\cell {}2{}{}{}{}{}8{}}|{\cell {}2{}{}{}{}{}8{}}|9|.
|2|9|6|7|3|8|{\cell 1{}{}{}5{}{}{}{}}|4|{\cell 1{}{}{}5{}{}{}{}}|.
\end{sudoku}
\caption{Example in Hidden Quad}
\end{figure}

In the example here, candidates 1, 4, 6 and 7 are the only 4 possibilities could still stay in these four cells, all the others will no longer existed after this technique has applied.

\section{X-wing}
\label{sec:X-wing}
It is an advanced technique to apply on when 4 cells are fully symmetric and have 2 candidates for a given digit in the symmetric cells of parallel rows. All other candidates for that digit from these 2 columns could be eliminated. On the other hand, all other candidates for that digit from these 2 rows could be removed if the situation is reversed.

\begin{figure}
\begin{sudoku}
   | |*| | | | | |*| |.
   | |X| | | | | |X| |.
   | |*| | | | | |*| |.
   | |*| | | | | |*| |.
   | |*| | | | | |*| |.
   | |*| | | | | |*| |.
   | |*| | | | | |*| |.
   | |X| | | | | |X| |.
   | |*| | | | | |*| |.
\end{sudoku}
\caption{Example in X-wing}
\end{figure}
For example here, two green X represent 2 given candidates in the parallel columns, if the diagram is symmetric, then all the cells contain these 2 candidates in the rows which intersect the parallel columns could have that 2 candidates eliminated.

\section{XY-wing}
\label{sec:XY-wing}

\begin{sudoku}
\label{xywing}
|{\cell {}{}{}{}{}{}{}{}{}}|{\cell {}{}{}{}{}{}{}{}{}}|{\cell {}{}{}{}{}{}{}{}{}}|{\cell {}{}{}{}{}{}{}{}{}}|{\cell {}{}{}{}{}{}{}{}{}}|{\cell {}{}{}{}{}{}{}{}{}}|{\cell {}{}{}{}{}{}{}{}{}}|{\cell {}{}{}{}{}{}{}{}{}}|{\cell {}{}{}{}{}{}{}{}{}}|.
|{\cell {}{}{}{}{}{}{}{}{}}|{\cell {}{}{}{}{}{}{}{}{}}|{\cell {}{}{}{}{}{}{}{}{}}|{\cell {}{}{}{}{}{}{}{}{}}|{\cell {}{}{}{}{}{}{}{}{}}|{\cell {}{}{}{}{}{}{}{}{}}|{\cell {}{}{}{}{}{}{}{}{}}|{\cell {}{}{}{}{}{}{}{}{}}|{\cell {}{}{}{}{}{}{}{}{}}|.
|{\cell {}{}{}{}{}{}{}{}{}}|{\cell {}{}{}{}{}{}{}{}{}}|{\cell {}{}{}{}{}{}{}{}{}}|{\cell {}{}{}{}{}{}{}{}{}}|{\cell {}{}{}{}{}{}{}{}{}}|{\cell {}{}{}{}{}{}{}{}{}}|{\cell {}{}{}{}{}{}{}{}{}}|{\cell {}{}{}{}{}{}{}{}{}}|{\cell {}{}{}{}{}{}{}{}{}}|.
|{\cell {}{}{}{}{}{}{}{}{}}|{\cell {}{}{}{}{}{}{}{}{}}|{\cell {}{}{}{}{}{}{}{}{}}|{\cell {}{}{}{}{}{}{}{}{}}|{\cell {}{}{}{}{}{}{}{}{}}|{\cell {}{}{}{}{}{}{}{}{}}|{\cell {}{}{}{}{}{}{}{}{}}|{\cell {}{}{}{}{}{}{}{}{}}|{\cell {}{}{}{}{}{}{}{}{}}|.
|{\cell {}{}{}{}{}{}{}{}{}}|{\cell {}{}{}{}{}{}{}{}{}}|{\cell {}{}{}{}{}{}{}{}{}}|{\cell {}{}{}{}{}{}{}{}{}}|{\cell {}{}{}{}{}{}{}{}{}}|{\cell {}{}{}{}{}{}{}{}{}}|{\cell {}{}{}{}{}{}{}{}{}}|{\cell {}{}{}{}{}{}{}{}{}}|{\cell {}{}{}{}{}{}{}{}{}}|.
|{\cell {}{}{}{}{}{}{}{}{}}|{\cell {}{}{}{}{}{}{}{}{}}|{\cell {}{}{}{}{}{}{}{}{}}|{\cell {}{}{}{}{}{}{}{}{}}|{\cell {}{}{}{}{}{}{}{}{}}|{\cell {}{}{}{}{}{}{}{}{}}|{\cell {}{}{}{}{}{}{}{}{}}|{\cell {}{}{}{}{}{}{}{}{}}|{\cell {}{}{}{}{}{}{}{}{}}|.
|{\cell {}{}{}{}{}{}{}{}{}}|{\cell {}{}{}{}{}{}{}{}{}}|{\cell {}{}{}{}{}{}{}{}{}}|{\cell {}{}{}{}{}{}{}{}{}}|{\cell {}{}{}{}{}{}{}{}{}}|{\cell {}{}{}{}{}{}{}{}{}}|{\cell {}{}{}{}{}{}{}{}{}}|{\cell {}{}{}{}{}{}{}{}{}}|{\cell {}{}{}{}{}{}{}{}{}}|.
|{\cell {}{}{}{}{}{}{}{}{}}|{\cell {}{}{}{}{}{}{}{}{}}|{\cell {}{}{}{}{}{}{}{}{}}|{\cell {}{}{}{}{}{}{}{}{}}|{\cell {}{}{}{}{}{}{}{}{}}|{\cell {}{}{}{}{}{}{}{}{}}|{\cell {}{}{}{}{}{}{}{}{}}|{\cell {}{}{}{}{}{}{}{}{}}|{\cell {}{}{}{}{}{}{}{}{}}|.
|{\cell {}{}{}{}{}{}{}{}{}}|{\cell {}{}{}{}{}{}{}{}{}}|{\cell {}{}{}{}{}{}{}{}{}}|{\cell {}{}{}{}{}{}{}{}{}}|{\cell {}{}{}{}{}{}{}{}{}}|{\cell {}{}{}{}{}{}{}{}{}}|{\cell {}{}{}{}{}{}{}{}{}}|{\cell {}{}{}{}{}{}{}{}{}}|{\cell {}{}{}{}{}{}{}{}{}}|.
\end{sudoku}

\section{XYZ-wing}
\label{sec:XYZ-wing}


\section{WXYZ-wing}
\label{sec:WXYZ-wing}


\section{Swordfish}
\label{sec:Swordfish}
Swordfish is an extended approach for X-wing which has really rare opportunity to apply on. It uses when all candidates for specific digit X are occurred in only three columns, all candidates for digit X not in these three columns could be eliminated. Each defined rows could have 2 or 3 candidates for specific digit X.

\begin{figure}
\begin{sudoku}
\label{swordfish}
   | |*| | |*| | |*| |.
   | |X| | |X| | |X| |.
   | |*| | |*| | |*| |.
   | |*| | |*| | |*| |.
   | |X| | |X| | |X| |.
   | |*| | |*| | |*| |.
   | |*| | |*| | |*| |.
   | |X| | |X| | |X| |.
   | |*| | |*| | |*| |.
\end{sudoku}
\caption{Swordfish: 3 rows in 3 columns}
\end{figure}

\begin{figure}
\begin{sudoku}
\label{swordfish}
   | |*| | |*| | |*| |.
   | | | | |X| | |X| |.
   | |*| | |*| | |*| |.
   | |*| | |*| | |*| |.
   | |X| | |X| | | | |.
   | |*| | |*| | |*| |.
   | |*| | |*| | |*| |.
   | |X| | | | | |X| |.
   | |*| | |*| | |*| |.
\end{sudoku}
\caption{Swordfish: 2 rows in 3 columns}
\end{figure}


\section{Intersection Removal}
\label{sec:Intersection Removal}
Sometimes, only hidden single and naked single could be able to deal all the Sudoku problems. When more difficult problems have been met, player will need to have more skill to support.
The intersection removal technique provides player to eliminate the impossibilities out of the row, or the column which related to the object block.
Briefly, the digit which is being concentrated on is only occurred in the same row or same column of the object block. Extending the view to complete row (or complete column) linked of whole block to eliminate all its possibilities in other blocks.

\begin{figure}
\begin{sudoku}
\label{intersectionremoval}
|2|{\cell {}{}{}{}56789}|{\cell {}{}{}{}5{}78{}}|{\cell {}{}3456{}{}9}|{\cell {}{}345{}789}|{\cell {}{}{}456789}|1|{\cell {}{}{}4{}67{}{}}|{\cell {}{}{}{}{}67{}9}|.
|1|4|{\cell {}{}{}{}5{}7{}{}}|{\cell {}{}{}{}56{}{}9}|2|{\cell {}{}{}{}567{}9}|{\cell {}{}{}{}{}67{}9}|8|3|.
|{\cell {}{}{}{}{}67{}9}|{\cell {}{}{}{}{}6789}|3|{\cell {}{}{}4{}6{}{}9}|1|{\cell {}{}{}4{}6789}|5|{\cell {}2{}4{}67{}{}}|{\cell {}2{}{}{}67{}9}|.
|{\cell {}{}345{}7{}{}}|{\cell 123{}5{}7{}{}}|{\cell {}2{}45{}7{}{}}|{\cell {}{}3456{}{}9}|{\cell {}{}345{}{}89}|{\cell {}{}{}456{}89}|{\cell {}{}{}4{}6789}|{\cell 12{}4{}67{}{}}|{\cell 12{}{}{}6789}|.
|{\cell {}{}34{}{}{}{}{}}|{\cell 123{}{}{}{}{}{}}|6|7|{\cell {}{}34{}{}{}89}|{\cell {}{}{}4{}{}{}89}|{\cell {}{}{}4{}{}{}89}|5|{\cell 12{}{}{}{}{}89}|.
|8|{\cell {}{}{}{}5{}7{}{}}|9|2|{\cell {}{}{}45{}{}{}{}}|1|3|{\cell {}{}{}4{}67{}{}}|{\cell {}{}{}{}{}67{}{}}|.
|{\cell {}{}{}4567{}9}|{\cell {}{}{}{}56789}|{\cell {}{}{}45{}78{}}|{\cell {}{}{}45{}{}{}9}|{\cell {}{}{}45{}7{}9}|3|2|{\cell 1{}{}{}{}67{}{}}|{\cell 1{}{}{}5678{}}|.
|{\cell {}{}{}4567{}9}|{\cell {}{}{}{}567{}9}|1|8|{\cell {}{}{}45{}7{}9}|2|{\cell {}{}{}{}{}67{}{}}|3|{\cell {}{}{}{}567{}{}}|.
|{\cell {}{}3{}5{}7{}{}}|{\cell {}23{}5{}78{}}|{\cell {}2{}{}5{}78{}}|1|6|{\cell {}{}{}{}5{}7{}{}}|{\cell {}{}{}{}{}{}78{}}|9|4|.
\end{sudoku}
\caption{Example in Intersection Removal}
\end{figure}
For example, candidate 4 (marked in yellow) has appeared twice in the same column of the top right block. Therefore, all candidate 4 occurrences (marked in pink) in the linked column will be removed.


\chapter{Conclusion}
\label{sec:Conclusion}
The Sudoku puzzle might be solved by only one technique in some simple case Nevertheless, usually two or more are commonly needed in dealing general problem such like using both naked single and hidden single. The number of techniques used will increase according to the difficulty of the puzzle. The harder problem needs more complex technique to support it.
These special approaches for solving puzzle are simple and in fact popularly used in general problem solving. They are efficient to support player to eliminate the impossibilities in the cells, and leave the rest possibilities to get to the final solution.


\bibliographystyle{plain}

\bibliography{Sudoku}

\end{document}
